\documentclass[a4paper]{article}

\usepackage[LGR]{fontenc}
\usepackage{titling}
\usepackage[greek]{babel}
\usepackage{amsmath}
\usepackage{hyperref}
\hypersetup{
    colorlinks,
    citecolor=black,
    filecolor=black,
    linkcolor=black,
    urlcolor=black
}

%------------------------------------------------------------------------------------------------
% basic info (date, title etc...)
%------------------------------------------------------------------------------------------------
\author{Παύλος Ορφανίδης \and Γιώργος Χατζηλίγος \and Σπύρος Κοντάκης}
\date{\today}
\title{Υπολογιστικά Μαθηματικά 2021--2022}

%------------------------------------------------------------------------------------------------
% Adding the option to append a number next to equations
%------------------------------------------------------------------------------------------------
\newcommand{\dd}[1]{\mathrm{d}#1}

\begin{document}
%------------------------------------------------------------------------------------------------
% Creating the title and the contents page
%------------------------------------------------------------------------------------------------
    \maketitle
    \tableofcontents

%------------------------------------------------------------------------------------------------
% Creating a first section that includes general data given by the instructor
%------------------------------------------------------------------------------------------------

    \section*{Γενικά δεδομένα}
        \begin{equation}
            AM = 4835
        \end{equation}
        \begin{equation}
            ms'' = (f_1+f_2)-b_s\rvert s' \lvert s'
        \end{equation}
        \begin{equation}
            I_z\omega '=\frac{d}{2}(f_2-f_1)-b_{\theta}\rvert\omega\lvert\omega
        \end{equation}
        \begin{equation}
            s(0)=s_0
        \end{equation}
        \begin{equation}
            s'(0)=0,\quad \omega(0)=0
        \end{equation}
        \[m=9kg\]
        \[d=1m\]
        \[I_z=0.38 kgm^2\]
%------------------------------------------------------------------------------------------------
% Excersise 1
%------------------------------------------------------------------------------------------------
    \section{Πρόβλημα 1}
        \subsection{Nα βρεθούν οι τύποι για την επίλυση του $\Pi.A.T$ με την Mέθοδο του $Euler$ και την βελτιωμένη μέθοδο του $Euler$ με τις παρακάτω τιμές για τις εισόδους και τις αρχικές συνθήκες}
        
        
        \subsubsection*{$Euler$ $s'$}
        Έχουμε από τα δεδομένα ότι:
        \begin{equation}
            s''=f'(t,s')=(f1+f2)-bs|s'|s'
        \end{equation}
        \begin{equation}
            s'=f(t,s)
        \end{equation}
        \[{[f_1,f_2]}^T={[A.M./7000, A.M./7000]}^T\]
        \[{[f_1,f_2]}^T={[A.M./7000, A.M./8000]}^T\]
        \[s_0=A.M./1000\]
        \[\theta_0=0\]

        
            \begin{tabular}{ll}
                Εφαρμόζουμε την μέθοδο $Euler$:                             &                               \\
                $t_n=t_0+nh$                                                & $s'_{n+1}=s'_n+hf'(t,s')_n$   \\
                το οποίο σημαίνει ότι:                                      & $s'_1=s'_0+hs''_0$            \\
                $t_1=t_0+1h$                                                & $s'_2=s'_1+hs''_1$            \\
                $t_2=t_0+2h$                                                & .                             \\
                .                                                           & .                             \\
                .                                                           & .                             \\
                .                                                           &                               \\
                $t_n=t_0+nh$                                                &                         
            \end{tabular}





        \subsubsection*{Bελτιωμένη μέθοδος $Euler$ $s'$}

        \begin{tabular}{ll}
            \multicolumn{2}{c}{Εφαρμόζουμε την βελτιωμένη μέθοδο $Euler$: }\\
            $t_n=t_0+nh$ & $s'_{n+1}=s'_n+\frac{h}{2}{[f'(t_n, s'_n)+f'(t_n+h, s'_n+hf'(t_n,s'_n))]}$\\
            το οποίο σημαίνει ότι:& \\
            $t_1=t_0+1h$          & $s'_n+\frac{h}{2}{[\frac{f_1+f_2-b_s\rvert s'_n \lvert s'}{m}+\frac{f_1+f_2}{m}-\frac{\lvert s'_n+h\frac{f_1+f_2-b_s\rvert s'_n \lvert s'}{m}\rvert}{m}\frac{(s'_n+h\frac{f_1+f_2-b_s\rvert s'_n \lvert s'}{m})}{m}]}$\\
            $t_2=t_0+2h$          & \\
            .                     & \\
            .                     & \\
            .                     & \\
            $t_n=t_0+nh$ & \\
        \end{tabular}
        
        
        
        
        \subsubsection*{Bελτιωμένη μέθοδος $Euler$ $s$}
        \begin{tabular}{ll}
            \multicolumn{2}{c}{Εφαρμόζουμε την βελτιωμένη μέθοδο $Euler$: }\\
            $t_n=t_0+nh$ & $s_{n+1}=s_n+hf(t,s)n$\\
            το οποίο σημαίνει ότι: & \\
            $t_1=t_0+1h$ & $s_{n+1}=s_n+hs'_n$\\
            $t_2=t_0+2h$ & $s_1=s_0+hs'_0$\\
            . & \\
            . & \\
            . & \\
            $t_n=t_0+nh$ & \\
        \end{tabular}

%------------------------------------------------------------------------------------------------
% Excersise 1γ
%------------------------------------------------------------------------------------------------
        \subsection{Eρώτημα γ: Μέθοδος $Euler$}

        \subsubsection{Δεδομένα:}
            \[f_1 + f_2 = Kps (sdes - s) - Kds (s')\]
            \[K_{ps} = 5\]
            \[K_{ds} = 15 + (AM/ 100)\]
            \[S_0 =0\]
            \[S_{des} = AM / 200\]

        \subsection{Μεταφορική Κίνηση}

        \[s' = - [ [ (f_1 + f_2 ) - Kps (s_{des} - s) ]/ K_{ds} ] = f(t,s)\]

        Άρα, για την συνάρτηση $s(t)$ έχουμε:
        
        \begin{tabular}{ll}
            $t_n = t_0 + nh$		&			$s_{n+1} = s_n  + hs'_n$ \\
            $t_1 = t_0  + 1h$			&		   $s_1    = s_0  + hs'_0$ \\
            $t_2 = t_0  + 2h$			&		   $s_2    = s_1  + hs'_1$ \\
            \multicolumn{2}{c}{.}\\
            \multicolumn{2}{c}{.}\\
            \multicolumn{2}{c}{.}\\
            $t_{30.000} = t_0 + 30.000h$&			    $s30.000 = s29.999 + hs'29.999$\\
            \multicolumn{2}{l}{Για την συνάρτηση $s'(t)$:}\\
            \multicolumn{2}{c}{$S''	= K_{ps}(s_{des} - s) - K_{ds}(s') - b_s|s'|s'$}\\
            \multicolumn{2}{l}{Άρα, προκύπτει:}\\
            $t_n = t_0 + nh$		&			$s'_{n+1} = s''_n  + hs''_n$\\
            $t_1 = t_0  + 1h$			&		   $s'_1    = s'_0  + hs''_0$\\
            $t_2 = t_0  + 2h$			&		   $s_2    = s'_1  + hs''_1$\\
            \multicolumn{2}{c}{.}\\
            \multicolumn{2}{c}{.}\\
            \multicolumn{2}{c}{.}\\
            $t_{30.000} = t_0 + 30.000h$&			   $s_{30.000} = s_{29.999} + hs''29.999$
        \end{tabular}


% Βελτιωμένη μέθοδος Euler



        \subsection{Πρόβλημα 1γ: Βελτιωμένη Μέθοδος Euler}

        \subsubsection{Δεδομένα}

        \[f_1 + f_2 = K_{ps}(s_{des} - s) - K_{ds}(s')\]
        \[K_{ps} = 5\]
        \[K_{ds} = 15 + (AM/ 100)\]
        \[S_0 =0\]
        \[S_{des} = AM / 200\]

        \subsubsection{Μεταφορική Κίνηση}   
            Για την s(t):

        tn = t0 + nh					sn+1 = sn  + hs’n

        t1 = t0  + 1h					   s1    = s0  + hs’0
        t2 = t0  + 2h					   s2    = s1  + hs’1
            .						     .
            .						     .
            .						     .
        t30.000 = t0 + 30.000h			     s30.000 = s29.999 + hs’29.999




        sn+1 = sn + (h/2) * [ f (tn, sn) + f (tn + h, sn + h (  f (  tn,sn ))]
                = sn  + (h/2) * [ f (tn, sn) + [ - (t1 + t2) – Kps ( sdes – ( sn + h([ - (f1 + f2 ) - Kps*              
                                                                                                * (sdes – sn) ]/ m))]|Kds

        Για τη s’(t):
        s’n+1 = s’n + (h/2) * [f’(tn, sn) + f (tn + h, s’n + h (  f’ (  tn,sn ))]]

        Οπότε,
        S’n  + (h/2) * [  f’  (tn, sn) + [ [(f1 + f2) – bs (s’n + h  |  (f1 + f2) – bs|s’n |s’n]  /  m  |  [ S’n + h (f1 + f2)  - bs |s’n|s’n ] / m  |  m 
\end{document}