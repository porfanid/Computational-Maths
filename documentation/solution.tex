\documentclass[a4paper]{article}

\usepackage[LGR]{fontenc}
\usepackage{titling}
\usepackage[greek]{babel}
\usepackage{amsmath}

\author{Παύλος Ορφανίδης \and Γιώργος Χατζηλίγος \and Σπύρος Κοντάκης}
\date{\today}
\title{Υπολογιστικά Μαθηματικά 2021--2022}

\newcommand{\dd}[1]{\mathrm{d}#1}

\begin{document}
    \maketitle
    \tableofcontents
    \section*{Γενικά δεδομένα}
        \begin{equation}
            ms'' = (f_1+f_2)-b_s\rvert s' \lvert s'
        \end{equation}
        \begin{equation}
            I_z\omega '=\frac{d}{2}(f_2-f_1)-b_{\theta}\rvert\omega\lvert\omega
        \end{equation}
        \begin{equation}
            s(0)=s_0
        \end{equation}
        \begin{equation}
            s'(0)=0,\quad \omega(0)=0
        \end{equation}
        \[m=9kg\]
        \[d=1m\]
        \[I_z=0.38 kgm^2\]
% -------------------------------------------------------------
    \section{Πρόβλημα 1}
        \subsection{Nα βρεθούν οι τύποι για την επίλυση του $\Pi.A.T$ με την Mέθοδο του $Euler$ και την βελτιωμένη μέθοδο του $Euler$ με τις παρακάτω τιμές για τις εισόδους και τις αρχικές συνθήκες}
        \subsubsection*{$Euler$ $s'$}
        Έχουμε από τα δεδομένα ότι:
        \begin{equation}
            s''=f'(x,y)=(f1+f2)-bs|s'|s'
        \end{equation}
        \begin{equation}
            s'=f(x,y)
        \end{equation}
        \[{[f_1,f_2]}^T={[A.M./7000, A.M./7000]}^T\]
        \[{[f_1,f_2]}^T={[A.M./7000, A.M./8000]}^T\]
        \[s_0=A.M./1000\]
        \[\theta_0=0\]

        Εφαρμόζουμε την μέθοδο $Euler$:
        \[t_n=t_0+nh\]
        το οποίο σημαίνει ότι:
        \[t_1=t_0+1h\]
        \[t_2=t_0+2h\]
        \[.\]
        \[.\]
        \[.\]
        \[t_n=t_0+nh\]

        \hrulefill{}

        \[s'_{n+1}=s'_n+hf'(t,y)n\]
        To οποίο σημαίνει ότι:
        \[s'_1=s'_0+hs''_0\]
        \[s'_2=s'_1+hs''_1\]
        \[.\]
        \[.\]
        \[.\]
        
        \subsubsection*{Bελτιωμένη μέθοδος $Euler$ $s'$}
        Εφαρμόζουμε την βελτιωμένη μέθοδο $Euler$:
        \[t_n=t_0+nh\]
        το οποίο σημαίνει ότι:
        \[t_1=t_0+1h\]
        \[t_2=t_0+2h\]
        \[.\]
        \[.\]
        \[.\]
        \[t_n=t_0+nh\]

        \hrulefill{}
        \[s'_{n+1}=s'_n+\frac{h}{2}{[f'(t_n, s'_n)+f'(t_n+h, s'_n+hf'(t_n,s'_n))]}\]
        'Ara
        \[s'_n+\frac{h}{2}{[\frac{f_1+f_2-b_s\rvert s'_n \lvert s'}{m}+\frac{f_1+f_2}{m}-\frac{\lvert s'_n+h\frac{f_1+f_2-b_s\rvert s'_n \lvert s'}{m}\rvert}{m}\frac{(s'_n+h\frac{f_1+f_2-b_s\rvert s'_n \lvert s'}{m})}{m}]}\]
        
        \subsubsection*{$Euler$ $s$}
        Εφαρμόζουμε την μέθοδο $Euler$:
        \[t_n=t_0+nh\]
        το οποίο σημαίνει ότι:
        \[t_1=t_0+1h\]
        \[t_2=t_0+2h\]
        \[.\]
        \[.\]
        \[.\]
        \[t_n=t_0+nh\]

        \hrulefill{}

        \[s_{n+1}=s_n+hf(t,s)n\]
        το οποίο σημαίνει ότι:
        \[s_{n+1}=s_n+hs'_n\]
        \[s_1=s_0+hs'_0\]

        \subsection{ερώτημα 1}
\end{document}