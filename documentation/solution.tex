\documentclass[a4paper]{article}

\usepackage[LGR]{fontenc}
\usepackage{titling}
\usepackage[greek]{babel}
\usepackage{amsmath}
\usepackage{mathtools}
\usepackage{breqn}
\usepackage{hyperref}
\hypersetup{
    colorlinks,
    citecolor=black,
    filecolor=black,
    linkcolor=black,
    urlcolor=black
}




%------------------------------------------------------------------------------------------------
% basic info (date, title etc...)
%------------------------------------------------------------------------------------------------
\author{Παύλος Ορφανίδης \and Γιώργος Χατζηλίγος \and Σπύρος Κοντάκης}
\date{\today}
\title{Υπολογιστικά Μαθηματικά 2021--2022}

%------------------------------------------------------------------------------------------------
% Adding the option to append a number next to equations
%------------------------------------------------------------------------------------------------
\newcommand{\dd}[1]{\mathrm{d}#1}

\begin{document}
%------------------------------------------------------------------------------------------------
% Creating the title and the contents page
%------------------------------------------------------------------------------------------------
    \maketitle
    \tableofcontents

%------------------------------------------------------------------------------------------------
% Creating a first section that includes general data given by the instructor
%------------------------------------------------------------------------------------------------

    \section*{Γενικά δεδομένα}
        \begin{equation}
            AM = 4835
        \end{equation}
        \begin{equation}
            ms'' = (f_1+f_2)-b_s\rvert s' \lvert s'
        \end{equation}
        \begin{equation}
            I_z\omega '=\frac{d}{2}(f_2-f_1)-b_{\theta}\rvert\omega\lvert\omega
        \end{equation}
        \begin{equation}
            s(0)=s_0
        \end{equation}
        \begin{equation}
            s'(0)=0,\quad \omega(0)=0
        \end{equation}
        \[m=9kg\]
        \[d=1m\]
        \[I_z=0.38 kgm^2\]
%------------------------------------------------------------------------------------------------
% Excersise 1
%------------------------------------------------------------------------------------------------
    \section{Πρόβλημα 1}
        \subsection*{Μεταφορική κίνηση}
        \subsubsection*{$Euler$ $s'$}
        Έχουμε από τα δεδομένα ότι:
        \begin{equation}
            s''=f'(t,s')=(f1+f2)-bs|s'|s'
        \end{equation}
        \begin{equation}
            s'=f(t,s)
        \end{equation}
        \[{[f_1,f_2]}^T={[A.M./7000, A.M./7000]}^T\]
        \[{[f_1,f_2]}^T={[A.M./7000, A.M./8000]}^T\]
        \[s_0=\frac{A.M.}{1000}\]
        \[\theta_0=0\]

        
            \begin{tabular}{ll}
                Εφαρμόζουμε την μέθοδο $Euler$:                             &                               \\
                $t_n=t_0+nh$                                                & $s'_{n+1}=s'_n+hf'(t,s')_n$   \\
                το οποίο σημαίνει ότι:                                      & $s'_1=s'_0+hs''_0$            \\
                $t_1=t_0+1h$                                                & $s'_2=s'_1+hs''_1$            \\
                $t_2=t_0+2h$                                                & .                             \\
                .                                                           & .                             \\
                .                                                           & .                             \\
                .                                                           &                               \\
                $t_n=t_0+nh$                                                &                         
            \end{tabular}


            \subsection*{Στροφική κίνηση}
            
            \begin{equation}
                \omega'=\frac{\frac{d}{2}(f_2-f_1)-b\theta\rvert\omega\lvert\omega}{I_z}=f(t,\omega)
            \end{equation}
            \subsubsection*{$Euler$}
            \begin{tabular}{ll}
                $t_{n+1}=t_0+nh$ & $\omega_{n+1} = \omega_0+h\omega'h$\\
                $t_1 = t_0+1h$&$\omega_1 = \omega_0+h\omega'_0$\\
                $t_2=t_0+2h$ & $\omega_2=\omega_1+h\omega'_1$\\
                .&.\\
                .&.\\
                .&.\\
                $t_{30.000}=t_0+29.999h$&$\omega_{30.000}=\omega_{29.999}+h\omega'_{29.999}$\\
            \end{tabular}

        \subsubsection*{Bελτιωμένη μέθοδος $Euler$ $s'$}

        \begin{tabular}{l|l}
            \multicolumn{2}{c}{Εφαρμόζουμε την βελτιωμένη μέθοδο $Euler$: }\\
            $t_n=t_0+nh$ & $s'_{n+1}=s'_n+\frac{h}{2}{[f'(t_n, s'_n)+f'(t_n+h, s'_n+hf'(t_n,s'_n))]}$\\
            το οποίο σημαίνει ότι:& $s'_n+\frac{h}{2}{[\frac{f_1+f_2-b_s\rvert s'_n \lvert s'}{m}+\frac{f_1+f_2}{m}-\frac{\lvert s'_n+h\frac{f_1+f_2-b_s\rvert s'_n \lvert s'}{m}\rvert}{m}\frac{(s'_n+h\frac{f_1+f_2-b_s\rvert s'_n \lvert s'}{m})}{m}]}$\\
            $t_1=t_0+1h$          &το οποίο σημαίνει ότι: \\
            $t_2=t_0+2h$          & $s'_1=s'_0+\frac{h}{2}{[\frac{f_1+f_2-b_s\rvert s'_0 \lvert s'}{m}+\frac{f_1+f_2}{m}-\frac{\lvert s'_0+h\frac{f_1+f_2-b_s\rvert s'_0 \lvert s'}{m}\rvert}{m}\frac{(s'_0+h\frac{f_1+f_2-b_s\rvert s'_0 \lvert s'}{m})}{m}]}$\\
            .                     & .\\
            .                     & .\\
            .                     & .\\
            $t_n=t_0+nh$          & $s'_n+\frac{h}{2}{[\frac{f_1+f_2-b_s\rvert s'_n \lvert s'}{m}+\frac{f_1+f_2}{m}-\frac{\lvert s'_n+h\frac{f_1+f_2-b_s\rvert s'_n \lvert s'}{m}\rvert}{m}\frac{(s'_n+h\frac{f_1+f_2-b_s\rvert s'_n \lvert s'}{m})}{m}]}$ \\
        \end{tabular}
        
        
      %--------------------------------------------------------------------  
        % Changes required
      %--------------------------------------------------------------------
        \subsubsection*{Bελτιωμένη μέθοδος $Euler$ $s$}
        \begin{tabular}{ll}
            \multicolumn{2}{c}{Εφαρμόζουμε την βελτιωμένη μέθοδο $Euler$: }\\
            $t_n=t_0+nh$ & $s_{n+1}=s_n+hf(t,s)n$\\
            το οποίο σημαίνει ότι: & \\
            $t_1=t_0+1h$ & $s_{n+1}=s_n+hs'_n$\\
            $t_2=t_0+2h$ & $s_1=s_0+hs'_0$\\
            . & \\
            . & \\
            . & \\
            $t_n=t_0+nh$ & \\
        \end{tabular}


        \subsection*{Στροφική κίνηση}

        \begin{dmath}
            \omega_{n+1}=\omega_n+\frac{h}{2}{[f(t, \omega)+f(t_n+h, \omega_n+f(t,\omega))]}=\omega_n+\frac{h}{2}{[\omega'_n+\frac{(\frac{d}{2}(f_2-f_1)-b\theta \lvert\omega_n+\omega'_n\rvert(\omega_n+\omega'_n))}{I_z}]}
        \end{dmath}
%------------------------------------------------------------------------------------------------
% Excersise 1γ
%------------------------------------------------------------------------------------------------
        \subsection{Eρώτημα γ: Μέθοδος $Euler$}

        \subsubsection{Δεδομένα:}
            \[f_1 + f_2 = K_{ps} (s_{des} - s) - K_{ds}(s')\]
            \[K_{ps} = 5\]
            \[K_{ds} = 15 + \frac{AM}{100}\]
            \[S_0 =0\]
            \[S_{des} = \frac{AM}{200}\]

        \subsection{Μεταφορική Κίνηση}
        \subsection{M'ejodos $Euler$}
        \begin{equation}
            f_1+f_2=K_{ps}(s_{des}-s)-K_{ds}s'
            \label{2}
        \end{equation}

        εφόσων ξέρω τον τύπο:

        \begin{equation}
            s''=\frac{f_1+f_2-b_s\rvert s'\lvert s'}{m}
            \label{1}
        \end{equation}

        \[(\ref{1}) \xRightarrow{(\ref{2})}s''=\frac{k_{ps}(s_{des}-s)-K_{ds}s'-b_s\rvert s'\lvert s'}{m}=f(t,s,s')\]

        \begin{tabular}{l|l}
            \multicolumn{2}{c}{Εφαρμόζουμε $Euler$ για την $s'$: }\\
            $t_n=t_0+nh$ & $s'_{n+1}=s'_n+hs''_n$\\
            το οποίο σημαίνει ότι: & \\
            $t_1=t_0+1h$ & $s'_1=s'_0+hs''_0$\\
            $t_2=t_0+2h$ & $s'_2=s'_1+hs''_1$(Dιότι έχει άγνωστη $s_1$)\\
            .            & $s'_3=s'_2+hs''_2$(Dιότι έχει άγνωστη $s_2$)\\
            . & \\
            . & \\
            $t_n=t_0+nh$ & \\
        \end{tabular}

        %--------------------------------------------
        %  Arrows to point between functions
        %--------------------------------------------
        
        \begin{tabular}{l|l}
            \multicolumn{2}{c}{Εφαρμόζουμε $Euler$ για την $s$: }\\
            $t_n=t_0+nh$ & $s_{n+1}=s_n+hs'_n$\\
            το οποίο σημαίνει ότι: & \\
            $t_1=t_0+1h$ & $s_1=s_0+hs'_0$\\
            $t_2=t_0+2h$ & $s_2=s_1+hs'_1$(Dιότι έχει άγνωστη $s_1$)\\
            .            &\\
            . & \\
            . & \\
            $t_n=t_0+nh$ & \\
        \end{tabular}
% Βελτιωμένη μέθοδος Euler



        \subsection{Πρόβλημα 1γ: Βελτιωμένη Μέθοδος Euler}

        \subsubsection{Δεδομένα}

        \[f_1 + f_2 = K_{ps}(s_{des} - s) - K_{ds}(s')\]
        \[K_{ps} = 5\]
        \[K_{ds} = 15 + (AM/ 100)\]
        \[S_0 =0\]
        \[S_{des} = AM / 200\]

        \subsubsection{Μεταφορική Κίνηση}   
        Για την $s(t)$:

        \begin{tabular}{ll}
            $t_n = t_0 + nh$		&			$s_{n+1} = s_n  + hs'_n$\\
            $t_1 = t_0  + 1h$		&			$s_1    = s_0  + hs'_0$\\
            $t_2 = t_0  + 2h$		&			$s_2    = s_1  + hs'_1$\\
            \multicolumn{2}{c}{.}\\
            \multicolumn{2}{c}{.}\\
            \multicolumn{2}{c}{.}\\
            $t_{30.000} = t_0 + 30.000h$&			     $s_{30.000} = s_{29.999} + hs'_{29.999}$
        \end{tabular}



        \begin{dmath}
            s_{n+1}=
                    s_n + \frac{h}{2}[f(t_n, s_n) + f(t_n + h, s_n + h (f(t_n,s_n ))]
                    =s_n+ \frac{h}{2}[f (t_n, s_n) + [- (t_1+t_2)-K_{ps} (s_{des}- (s_n + h (\frac{- (f_1 + f_2)-K_{ps} (s_{des}-s_n)}{m}))]|K_{ds}
        \end{dmath}

        Για τη $s'(t)$:

        \[s'_{n+1} = s'_n + \frac{h}{2}[f'(t_n, s_n) + f (t_n + h, s'_n + h (  f' (t_n,s_n )))]\]

        Οπότε,

        \[s'_n  + (h/2)(f'(t_n, s_n) + \frac{(f_1 + f_2) - b_s (s'_n + h  \lvert  (f_1 + f_2) - b_s\rvert s'_n |s'_n}{m}  \lvert  \frac{S'_n + h (f_1 + f_2)  - b_s |s'_n|s'_n}{m}  \rvert  m) \]
\end{document}